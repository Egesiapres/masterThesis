\selectlanguage{italian}

\chapter{Inquadramento contestuale}
Il primo capito ha lo scopo di contestualizzare il progetto oggetto dell'elaborato fornendone un'iniziale visione d'insieme. Per farlo va a trattarne le caratteristiche di più alto livello, la cui descrizione è doverosa prima di addentrarsi in questioni più teoriche o pratiche. A tal fine, il capitolo incomincia offrendo una panoramica generale dell'area tematica a cui il progetto appartiene, la telemedicina, e del suo substrato aziendale, Link-Up e Cerba HealthCare, per poi proseguire con una sua disamina complessiva che tiene conto degli obiettivi perseguiti nel corso dell'attività di sviluppo.

\section{Telemedicina}
Nonostante il mondo in cui viviamo sia caratterizzato da una forte presenza tecnologica in quasi tutti i settori aziendali, alcuni di essi stanno attraversando un'autentica fase di transazione proprio al giorno d'oggi, con scenari che variano da paese a paese. Generalmente, questo tipo di settori o sono protagonisti di una radicale rivoluzione, la quale richiede un intervento diffuso con l'intento di ammodernare dal profondo, oppure hanno già beneficiato dell'impatto della tecnologia, ma solo parzialmente, necessitando quindi di proseguire nel processo di innovazione.

Uno dei settori più ambivalenti, in questo senso, è quello medico: infatti, seppur in alcuni frangenti possa vantare uno dei più alti livelli di modernità, in altri questo campo rimane ancorato a metodologie del passato, specialmente in Italia.\\
Tuttavia, negli ultimi anni, i tempi sembrano maturi per un importante balzo in avanti. Uno dei fattori che sicuramente ha più contribuito a fornire un decisivo impulso evolutivo è l'emergenza sanitaria causata dal COVID-19: infatti, le caratteristiche del virus e la sua enorme diffusione,  sono riusciti a mettere in crisi il settore medico mostrandone chiaramente i limiti attuali, e facendo insorgere l'urgenza di nuovi strumenti.

L'accelerazione dovuta alla pandemia ha portato all'ingresso\footnote{La Telemedicina ha fatto il suo ingresso nel \acrshort{ssn} nel dicembre 2020. In altri paesi europei, fra cui Svezia, Norvegia, Regno Unito e Spagna è molto diffusa da tempo.} di una nuova soluzione medica nel \acrfull{ssn} italiano, ovvero della Telemedicina. Poiché essa rappresenta ancora un qualcosa di pionieristico e sperimentale, e, di conseguenza, di avvolto da una fitta nube di confusione, è bene fare chiarezza.\\
La Telemedicina è definita come:
\begin{quote}
  «[...] una modalità di erogazione di servizi di assistenza sanitaria, tramite il ricorso a tecnologie innovative, in particolare alle \acrfull{ict}, in situazioni in cui il professionista della salute e il paziente [...] non si trovano nella stessa località. La Telemedicina comporta la trasmissione sicura di informazioni e dati di carattere medico [...] per la prevenzione, la diagnosi, il trattamento e il successivo controllo dei pazienti. I servizi di Telemedicina vanno assimilati a qualunque servizio sanitario diagnostico/terapeutico. Tuttavia la prestazione in Telemedicina non sostituisce la prestazione sanitaria tradizionale nel rapporto personale medico-paziente, ma la integra per potenzialmente migliorare efficacia, efficienza e appropriatezza»\cite{Stato_Telemedicina}.
\end{quote}
% spazio
% I principali benefici che tale strumento è in gr

% Controllo costante (situazione clinica/appropriatezza cura); 
% Diagnosi precoce di patologie; 
% Assistenza continua da remoto; 
% Intervento tempestivo in caso di criticità; 
% Riduzione dei costi (paziente/stato?).
% L'obiettivo della Telemedicina, e più in generale dei servizi di sanità di nuova generazione, è quindi quello di attuare un profondo rinnovamento del sistema sanitario erogando al paziente prestazioni di maggiore qualità, e e abbattendo i costi di gestione dall'altro.

\section{Link-Up e Cerba HealthCare}
Link-Up è un'azienda specializzata nello sviluppo di \textit{software serverless}, di tipo \textit{cloud-native}, e nella progettazione di soluzioni di \textit{Data Analytics} e Intelligenza Artificiale che sfruttano i dati ricavati da dispositivi eterogenei connessi all'\gls{iot}.
\\
Questa realtà opera prevalentemente nel settore medico-sanitario e offre ai suoi clienti strategie innovative e supporto orientato alla trasformazione digitale.

Uno dei contesti aziendali di maggiore rilievo con cui Link-Up sta collaborando in maniera attiva è Cerba HealthCare, un gruppo internazionale francese leader nell'ambito della diagnostica ambulatoriale che recentemente è stato protagonista di una significativa espansione in Italia con l'acquisizione di numerose strutture mediche.\\
La stimolante missione che Cerba sta affrontando consiste nell'offrire un'assistenza sanitaria digitalizzata capace di mettere al centro la persona, semplificando l'esperienza di pazienti, aziende e operatori tramite strumenti innovativi, come, ad esempio, cartelle cliniche elettroniche, servizi di telemedicina, prenotazioni online, etc...

\section{OMNIA}
\label{sec:OMNIA}
Ad oggi, Link-Up coopera con Cerba HealthCare per la realizzazione di un complesso applicativo per la digitalizzazione dei flussi operativi aziendali, che mira a incidere sulla \textit{performance} dei professionisti della sanità ammodernando i processi che li vedono coinvolti e agevolando le gestione delle attività e delle strutture sparse sul territorio italiano.

L'applicativo si configura come una piattaforma composta da moduli che digitalizzano i flussi e la gestione dei diversi verticali in cui è attiva Cerba HealthCare Italia. Ciascun modulo prevedere l'interazione di diverse figure professionali, anche in simultanea, e la possibilità di svolgere azioni differenti a seconda del ruolo ricoperto.\\
A partire da settembre 2022, il team di sviluppo di Link-Up è tornato a focalizzarsi su “OMNIA”, uno dei moduli più importanti della piattaforma di digitalizzazione. Il modulo “OMNIA” permette di gestire in maniera capillare le strutture di Cerba HealthCare Italia sul territorio, nonché digitalizzare i flussi dell'attività logistica a esse legata.\\
Nello specifico, sono state programmate numerose evolutive che riguardano per lo più l'aggiunta di nuove funzionalità richieste dal committente, in particolare a supporto del team di \textit{marketing}, così da rendere possibile attraverso il modulo “OMNIA” la gestione di contenuti presentati all'interno del sito web aziendale, con l'obiettivo di centralizzare e unificare la gestione delle informazioni utili agli utenti dei centri di Cerba HealthCare Italia.

“OMNIA”, così come gli altri moduli che compongono la piattaforma sviluppata da Link-Up, rappresenta un'articolata \acrlong{spa} la cui interfaccia utente è stata sviluppata tramite le librerie React e \gls{mui}.
% e il cui funzionamento poggia sui servizi offerti da \gls{aws}.

% (strutture, lab, \textit{Finance}, \textit{Users} e \textit{Marketing})
% (accettatore)
% (Area, Center e REO)
% (coordinatore turni e professionisti)
% (coordinatore prenotazioni)
% (agenzia e fornitore corsi)
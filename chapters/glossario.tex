% GLOSSARY
% c1
\newglossaryentry{iotg}{
  name={\acrshort*{iot}},
  description={L'\acrfull*{iot}, in italiano “internet delle cose”, è un paradigma tecnologico che ha come obiettivo la connessione di vari oggetti della nostra quotidianità a internet. Sfruttando la rete, i dispositivi riescono a trasmettere dati provenienti dai sensori di cui sono dotati},
  first={\acrfull*{iot}}
}
% \newglossaryentry{crmg}{
%   name={\acrshort*{crm}},
%   description={Un \acrfull*{crm} è un \textit{software} che ingloba al suo interno molteplici funzionalità per la gestione del rapporto con il cliente},
%   first={\acrfull*{crm}}
% }
% \newglossaryentry{awsg}{
%   name={\acrshort*{aws}},
%   description={Gli \acrfull*{aws} sono dei servizi di \textit{cloud computing} offerti da Amazon per aiutare le aziende e gli sviluppatori nella creazione di applicazioni flessibili e affidabili. Grazie a essi, ad esempio, è possibile elaborare dati, archiviare database e condividere contenuti},
%   first={\acrfull*{aws}}
% }

% c2
\newglossaryentry{uxg}{
  name={\acrshort*{ux}},
  description={L'\acrfull*{ux}, in italiano “esperienza utente”, è il rapporto che si instaura in seguito all'interazione di un individuo con lo strumento in uso, sia esso un prodotto o un servizio. All'interno dell'ampia gamma di fattori che contribuiscono alla \acrshort*{ux}, alcuni fra i più importanti sono l'utilità, la semplicità d'uso e la piacevolezza. La \acrshort*{ui} fa parte dell'esperienza utente},
  first={\acrfull*{ux}}
}
\newglossaryentry{htmlg}{
  name={\acrshort*{html}},
  description={L'\acrfull*{html} è un linguaggio di \textit{markup} che consente la gestione della struttura di base delle pagine web, ossia l'impaginazione e la formattazione di documenti ipertestuali. Poiché non fornisce vere e proprie istruzioni, non è considerato un linguaggio di programmazione. Nel 2014 è stata rilasciato l'\acrshort*{html}5, la sua attuale versione disponibile},
  first={\acrfull*{html}}
}
\newglossaryentry{httpg}{
  name={\acrshort*{http}},
  description={L'\acrfull*{http} è un protocollo di livello applicativo usato per la trasmissione di informazioni nel web. Esso opera all'interno di un'architettura client-server},
  first={\acrfull*{http}}
}
\newglossaryentry{seog}{
  name={\acrshort*{seo}},
  description={La \acrfull*{seo} è un insieme di pratiche volte a migliorare la scansione, l'indicizzazione e il posizionamento di informazioni e contenuti di un sito web. Ottimizzare la \acrshort*{seo} permette di migliorarne il posizionamento delle pagine nei risultati del motore di ricerca},
  first={\acrfull*{seo}}
}
\newglossaryentry{cssg}{
  name={\acrshort*{css}},
  description={Il \acrfull*{css} è un linguaggio che consente la gestione della veste grafica delle pagine web \acrshort*{html}. Nel 2011 è stata rilasciato il \acrshort*{css}3, la sua attuale versione disponibile},
  first={\acrfull*{css}}
}
\newglossaryentry{flashg}{
  name={\acrshort*{flash}},
  description={\acrlong*{flash} è stato un \textit{software} che per molto tempo ha permesso alle pagine web di essere dinamiche e interattive, ad esempio, riproducendo video. A causa del suo peso e dei problemi di sicurezza a cui esponeva, \acrshort*{flash} è stato nel corso degli anni lentamente sostituito da tecnologie più leggere e sicure (ad esempio, \acrshort*{html}5) fino a essere abbandonato completamente nel 2020},
  first={\acrlong*{flash}}
}
\newglossaryentry{ajaxg}{
  name={\acrshort*{ajax}},
  description={L'\acrfull*{ajax} è un'insieme di tecnologie di sviluppo \textit{software} create per una migliore gestione dell'interattività delle pagine web dinamiche. \acrshort*{ajax} consente un leggero disallineamento fra il momento in cui la richiesta viene inoltrata e quello in cui viene restituito il risultato. Il suo obiettivo è quello di far dialogare client e server in  \textit{background} per aggiornare il contenuto della pagina senza dover eseguire un \textit{refresh} manuale},
  first={\acrlong*{ajax}}
}
\newglossaryentry{usabg}{
  name={usabilità},
  description={L'\gls*{usabg}, in inglese \textit{usability}, viene definita dalla norma ISO come il «grado in cui un prodotto può essere usato da particolari utenti per raggiungere certi obiettivi con efficacia, efficienza e soddisfazione in uno specifico contesto d’uso»}
}
\newglossaryentry{respg}{
  name={responsività},
  description={La \gls*{respg}, in inglese \textit{responsivness}, è la capacità di un prodotto di adattarsi a diversi contesti d'utilizzo. Il termine viene usato per riferirsi ad applicazioni che, implementando un design \textit{responsive}, sono in grado di gestire la visualizzazione dei contenuti a seconda del dispositivo in uso},
  first={responsività}
}

% c4
\newglossaryentry{opensg}{
  name={\textit{open-source}},
  description={Il termine \gls*{opensg} viene utilizzato per riferirsi a \textit{software} liberamente manipolabili dagli utenti in quanto privi di \textit{copyright}},
}
\newglossaryentry{domg}{
  name={\acrshort*{dom}},
  description={Il \acrfull*{dom} rappresenta la struttura di una pagina web tramite una struttura ad albero. Ogni ramo dell'albero termina con un nodo e ogni nodo contiene oggetti. Il \acrshort*{dom} collega le pagine web agli script o ai linguaggi di programmazione},
  first={\acrfull*{dom}}
}
\newglossaryentry{urlg}{
  name={\acrshort*{url}},
  description={L'\acrfull*{url} rappresenta l'indirizzo di una risorsa (ad esempio, una pagina) all'interno del web; essa è immagazzinata all'interno di un server web, e vi si può accedere, lato client, tramite il browser},
  first={\acrfull*{url}}
}
\newglossaryentry{es6g}{
  name={\acrshort*{es6}},
  description={L'\acrfull*{es6} (anche noto come “ECMAScript 2015”) è la sesta versione di uno standard per i linguaggi di scripting atto al garantire il corretto funzionamento delle pagine web sui diversi tipi di browser. Con \acrshort*{es6}, nel 2015 JavaScript è stato profondamente aggiornato tramite l'introduzione di nuove funzionalità (ad esempio, grazie alle \textit{keyword} \texttt{let} e \texttt{const}, alla \textit{arrow function}, etc...)},
  first={\acrfull*{es6}}
}
% \newglossaryentry{apig}{
%   name={API},
%   description={Un \acrfull*{api} è uno particolare insieme di regole e specifiche che un \textit{software} deve seguire per accedere e usufruire di servizi e/o risorse offerte da un'altro \textit{software} che implementa quell'\acrshort*{api}},
%   first={\acrfull*{api}}
% }


% ACRONYMS
% c1
\newacronym{ssn}{SSN}{Servizio Sanitario Nazionale}
\newacronym{ict}{ICT}{\textit{Information and Communication Technologies}}

\newacronym[see={iotg}]{iot}{IoT}{\textit{Internet of Things}\glsadd{iotg}}
% \newacronym[see={crmg}]{crm}{CRM}{\textit{Customer Relationship Management}\glsadd{crmg}}
% \newacronym[see={awsg}]{aws}{AWS}{\textit{Amazon Web Services}\glsadd{awsg}}

% c2
\newacronym[see={uxg}]{ux}{UX}{\textit{User Experience}\glsadd{uxg}}

\newacronym{www}{WWW}{\textit{World Wide Web}}

\newacronym{cern}{CERN}{\textit{Conseil Européen pour la Recherche Nucléaire}}

\newacronym[see={urlg}]{url}{URL}{\textit{Uniform Resource Locator}\glsadd{urlg}}

% \newacronym{mvc}{MVC}{\textit{Model Control View}}
\newacronym{mpa}{MPA}{\textit{Multiple-Page Application}}
\newacronym{ssr}{SSR}{\textit{Server-Side Rendering}}
\newacronym{spa}{SPA}{\textit{Single-Page Application}}

\newacronym[see={htmlg}]{html}{HTML}{\textit{HyperText Markup Language}\glsadd{htmlg}}
\newacronym[see={httpg}]{http}{HTTP}{\textit{HyperText Transfer Protocol}\glsadd{httpg}}
\newacronym[see={seog}]{seo}{SEO}{\textit{Search Engine Optimization}\glsadd{seog}}
\newacronym[see={cssg}]{css}{CSS}{\textit{Cascading Style Sheets}\glsadd{cssg}}
\newacronym[see={flashg}]{flash}{\textit{Flash}}{\textit{Adobe Flash Player}\glsadd{flashg}}
\newacronym[see={ajaxg}]{ajax}{AJAX}{\textit{Asynchronous JavaScript and XML}\glsadd{ajaxg}}

\newacronym{ia}{IA}{\textit{Artificial Intelligence}}
\newacronym{db}{DB}{\textit{DataBase}}

\newacronym{pwa}{PWA}{\textit{Progressive Web Application}}
\newacronym{iwa}{IWA}{\textit{Isomorphic Web Application}}
\newacronym{gps}{GPS}{\textit{Global Positioning System}}
\newacronym{qr}{QR}{\textit{Quick Response}}
\newacronym{soa}{SOA}{\textit{Service Oriented Architecture}}

% c3
\newacronym{it}{IT}{\textit{Information Technologies}}
\newacronym{asd}{ASD}{\textit{Agile Software Development}}

% c4
\newacronym{ui}{UI}{\textit{User Interface}}

\newacronym[see={domg}]{dom}{DOM}{\textit{Document Object Model}\glsadd{domg}}

\newacronym{jsx}{JSX}{\textit{JavaScript Syntax Extension}}
\newacronym{dry}{DRY}{\textit{Don't Repeat Yourself}}

\newacronym[see={es6g}]{es6}{ES6}{\textit{ECMAScript 6}\glsadd{es6g}}

\newacronym{api}{API}{\textit{Application Programming Interface}}
% \newacronym[see={apig}]{api}{API}{\textit{Application Programming Interface}\glsadd{apig}}

% c5
\newacronym{mui}{MUI}{Material UI}
\newacronym{rest}{REST}{\textit{Respresentational State Transfer}}
\newacronym{soap}{SOAP}{\textit{Single Object Access Protocol}}
\newacronym{crud}{CRUD}{\textit{Create Read Update Delete}}
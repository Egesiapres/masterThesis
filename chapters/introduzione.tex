\selectlanguage{italian}

\chapter{Introduzione}
Ad oggi, in vari paesi del mondo fra cui l'Italia, i tempi sembrano maturi per un importante balzo in avanti di uno dei settori al contempo più rilevante e delicato per la vita delle persone, quello sanitario. Ciò è evidente non solo in virtù delle numerose possibilità attualmente offerte dalla tecnologia, ma anche per la necessità di poter fare affidamento su uno strumento profondamente strutturato e resiliente, capace di erogare servizi in modo intuitivo, e di fronteggiare efficacemente potenziali situazioni di crisi. In quest'ottica, una delle soluzioni identificate è la Telemedicina, un paradigma che mira alla digitalizzazione dei processi che coinvolgono pazienti e professionisti attraverso l'impiego delle tecnologie informatiche e della comunicazione.

Il progetto protagonista dell'elaborato in questione appartiene a questo contesto, e si configura come un'attività di sviluppo che ha portato all'implementazione di nuovi componenti all'interno dell'interfaccia utente dell'applicazione web “OMNIA”. La relazione rappresenta un'analisi della suddetta attività progettuale e l'esposizione delle diverse sezioni realizza una progressione: nelle prime due vengono affrontate le tematiche teorico/nozionistiche di alto livello; la terza e la quarta si rapportano con aspetti più pragmatici e tecnici di basso livello; infine, nella quinta viene fornita una trattazione di ciò che di pratico è stato svolto.

“OMNIA” rappresenta uno dei moduli di cui si compone l'articolata piattaforma per la digitalizzazione dei flussi operativi aziendali che Link-Up sta sviluppando per Cerba HealthCare, un gruppo francese specializzato in diagnostica ambulatoriale che mira a distinguersi anche in Italia per la qualità e la modernità dei servizi offerti in ambito medico-sanitario. Una più dettagliata descrizione della “cornice” del progetto, ovvero di tutti quegli aspetti che permettono una collocazione tematica dell'applicativo sul quale si è operato, è fornita all'interno del primo capitolo. 

Come anticipato, il modulo protagonista dell'elaborato, così il come la piattaforma nella sua interezza, rappresenta un'applicazione web, o meglio, una \acrlong*{spa}. Le web app di questo genere si avvalgono di un funzionamento che simula quello di una singola pagina web, e sono molto popolari grazie all'esperienza utente, alla rapidità di sviluppo e alla riduzione dei costi che sono in grado di offrire. Questi e altri concetti sono affrontati nel secondo capitolo, sezione atta alla descrizione delle caratteristiche che il prodotto realizzato per Cerba HealthCare presenta.
% posizionandole in una prospettiva storica e architetturale. 

L'attività di sviluppo condotta con il team di Link-Up è avvenuta tramite il lavoro da remoto, il quale è stato declinato secondo la metodologia agile in combinazione con il \textit{framework Scrum}. Con l'impiego di questo \textit{modus operandi}, i compiti previsti dalle evolutive programmate per “OMNIA” sono stati organizzati in \textit{sprint}, delle brevi finestre temporali dedicate allo sviluppo di una specifica funzionalità dell'applicativo, pensate per agevolare il lavoro degli sviluppatori e l'interazione con il committente. Le strategie adottate per la gestione della cooperazione con il team sono descritte nel terzo capito, sezione che si discosta dalle precedenti per \textit{focus} tematico, e costituisce un autentico spartiacque all'interno della struttura generale dell'elaborato. A partire da qui, il fulcro della trattazione diviene l'attività di sviluppo.

L'interfaccia utente dell'applicazione web “OMNIA” è stata realizzata per lo più con l'utilizzo delle librerie React e \acrlong*{mui}, due strumenti per lo sviluppo lato \textit{front-end} molto popolari nella community grazie a pregi come semplicità, supporto, velocità, flessibilità e modularità. In conseguenza a ciò, esse rappresentano le due soluzioni tecnologiche principali con cui mi sono relazionato in prima persona per la realizzazione delle implementazioni, nonché i due argomenti fondamentali trattati nel quarto capitolo. Questo prosegue nella descrizione, iniziata nella precedente sezione, delle soluzioni impiegate nel corso dell'attività di sviluppo, incentrandosi però sugli aspetti tecnologici.
% affrontando anche le librerie di minore importanza implementate e usate nel progetto.

Le diverse implementazioni realizzate sono state caratterizzate da un differente livello di complessità, ed hanno avuto a che fare, in particolare, con l'introduzione di una serie di sezioni a supporto del team di \textit{marketing}. Le \textit{feature} più importanti sviluppate sono state elencate all'interno del quinto e ultimo capitolo, sezione nella quale viene mostrato l'\textit{output} dell'attività di sviluppo condotta ai fini del progetto. Al suo interno viene anche descritto il “Flusso di sviluppo standard”, ovvero la procedura tipo seguita per realizzare un'implementazione, comprensiva di tutti gli \textit{step} necessari. L'idea di fornire un punto di riferimento stabile e allo stesso tempo un \textit{fil rouge} in grado di tracciare il percorso attraverso i vari passaggi seguiti, spesso solo parzialmente presenti, è stata motivata dal desiderio di generalizzare il più possibile il processo realizzativo prima di addentrarsi nell'analisi approfondita di ogni singolo caso.
\selectlanguage{italian}

\chapter{Conclusioni}
L'attività progettuale condotta ai fini dell'elaborato ha rappresentato un'esperienza di grande valore, che mi ha permesso di raggiungere diversi traguardi personali e di crescere professionalmente.

\textit{In primis}, ho potuto relazionarmi per la prima volta con il mondo del lavoro, nello specifico, interfacciandomi con una realtà disponibile e moderna come lo è Link-Up. Grazie all'inserimento in tale contesto aziendale, infatti, ho potuto sperimentare cosa significhi operare come sviluppatore \textit{front-end}, la figura professionale che ambisco a ricoprire in futuro, e far parte di un team interfunzionale in cui si collabora per far fronte a obiettivi comuni.\\
In secondo luogo, ho avuto l'opportunità di misurarmi con un progetto stimolante, collocato nell'innovativo e impattante contesto della Telemedicina. I vari mesi dedicati allo sviluppo delle implementazioni previste per il modulo “OMNIA”, di fatto, si sono tradotti da un lato in un applicativo aggiornato, che include le evolutive richieste dal committente, e che andrà a incidere positivamente sul flusso lavorativo di coloro che lo utilizzeranno quotidianamente, e dall'altro nell'oggetto protagonista della mia prova finale.

In aggiunta, nel corso dell'attività di sviluppo ho potuto accrescere le conoscenze in mio possesso attraverso i preziosi consigli ricevuti dai professionisti con i quali sono venuto a contatto, e applicare varie nozioni apprese tramite gli insegnamenti del corso di laurea.\\
Ancora una volta, va sottolineata l'importanza ricoperta delle librerie React e \acrlong*{mui}. Dato il fondamentale ruolo da esse ricoperto all'interno del progetto, con il loro impiego ho avuto l'opportunità di cimentarmi con due tecnologie estremamente al passo con i \textit{trend} attuali e di rilievo per gli sviluppatori \textit{front-end}. Il loro utilizzo ha suscitato in me il desidero di testare altre soluzioni che orbitano attorno al mondo di JavaScript e di React, nell'ottica di migliorare e accrescere l'attuale bagaglio di conoscenze. Ad ora, fra gli strumenti che più attirano il mio interesse vi sono React Native ed Electron, due \textit{framework} che permettono di sfruttare competenze di sviluppo web per la creazione di applicazioni \textit{mobile}. Ho avuto poi la possibilità di integrare all'interno della mia \textit{ruotine} lavorativa la metodologia agile di tipo \textit{Scrum}. Ciò mi ha consentito di familiarizzare con uno strumento largamente impiegato nell'ambito dello sviluppo del \textit{software}, e di rapportarmi con pregi e difetti di una strategia altamente dinamica, che, nel mio caso, ha previsto uno stretto contatto con il committente, con altri sviluppatori, e, alle volte, anche con il team di design.\\
Inoltre, ho potuto imbattermi in problemi e situazioni critiche caratteristiche di un'attività di sviluppo, nonché di un progetto così ambizioso. Infatti, è capitato di avere difficoltà nell'ambientarsi e nel destreggiarsi all'interno di un progetto già avviato, gestito da persone dotate di un maggior livello di competenza, nell'utilizzo di soluzioni tecnologiche che, in una certa misura, rappresentano una novità, oppure ancora nel dover adattare l'attività in svolgimento alle esigenze della prova finale. 

In conclusione, per quanto concerne il futuro dell'applicativo, il modulo “OMNIA” continuerà a essere gestito con le stesse modalità analizzate nel corso dell'elaborato. Per esso, seppur in misura minore rispetto al passato, sono programmate ulteriori evolutive che porteranno sia all'implementazione di nuove funzionalità per il supporto ad altri team, che alla loro ottimizzazione attraverso l'integrazione del \textit{feedback} ricevuto.\\
Invece, lo sviluppo della piattaforma per la digitalizzazione dei flussi di lavoro proseguirà perseguendo due obiettivi principali: per prima cosa, altri moduli che lo compongono, a rotazione (similmente a come è avvenuto per “OMNIA”), diverranno il fulcro dell'attività lavorativa del team per l'aggiunta di nuove funzioni; in secondo luogo, l'applicazione \textit{mobile}, esistente ma ancora in stato embrionale, verrà migliorata per avvicinarsi sempre di più all'essere uno strumento usabile e completo.